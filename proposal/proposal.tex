\documentclass[11pt]{article}
\bibliographystyle{siam}
\usepackage{listings}
\lstset{basicstyle=\ttfamily\footnotesize,breaklines=true}

\title{The Neural Basis of Loss Aversion in Decision-Making Under Risk}
\author{
  Jara, Jon\\
  \texttt{jonmigueljara}
  \and
  Sanderson, Will\\
  \texttt{wtsanderson}
  \and
  Shishido, Juan\\
  \texttt{juanshishido}
  \and
  Wu, Paul\\
  \texttt{yuanyiwu}
  \and
  Xu, Wendy\\
  \texttt{berithsu}
}

\begin{document}
\maketitle

We are interested in understanding the neural bases for decision making under
risk. Insights from psychology have led to the development of Prospect Theory,
which asserts that people evaluate outcomes in terms of gains and losses and
relative to some reference point\cite{kahnemantversky}. It uses a concept
called loss aversion to describe the preference toward avoiding losses to
acquiring equal-sized gains. Tom, Fox, Trepel, and Poldrak (2007) investigate
the neural sources of loss aversion. Their goal was to identify and isolate the
regions of the brain associated with evaluating gambles and deciding whether or
not to accept them.

For each of the 16 subjects---eight males and eight females with an average
age of 22---three trials of the ``mixed gambles task'' are performed. The
\lstinline{ds005} data set contains the blood-oxygen-level-dependent contrasts
for each subject and trial. We successfully loaded several
\lstinline{bold.nii.gz} files, which have a shape of (64, 64, 34, 240).

Briefly explain what approach you intend to take for exploring
the data and paper.  If you intend to ``reproduce'' some aspect of the paper,
explain in exactly what sense you mean to do this.  If you are thinking about
validating the data describe the assumpution used in the analysis and indicate
how you might check them.

\bibliography{proposal}

\end{document}
