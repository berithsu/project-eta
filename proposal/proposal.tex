\documentclass[11pt]{article}
\bibliographystyle{siam}
\usepackage{listings}
\lstset{basicstyle=\ttfamily\footnotesize,breaklines=true}

\title{The Neural Basis of Loss Aversion in Decision-Making Under Risk}
\author{
  Jara, Jon\\
  \texttt{jonmigueljara}
  \and
  Sanderson, Will\\
  \texttt{wtsanderson}
  \and
  Shishido, Juan\\
  \texttt{juanshishido}
  \and
  Wu, Paul\\
  \texttt{yuanyiwu}
  \and
  Xu, Wendy\\
  \texttt{berithsu}
}

\begin{document}
\maketitle

We are interested in understanding the neural bases for decision making under
risk. Insights from psychology have led to the development of Prospect Theory,
which asserts that people evaluate outcomes in terms of gains and losses and
relative to some reference point\cite{kahnemantversky}. It uses a concept
called loss aversion to describe the preference toward avoiding losses to
acquiring equal-sized gains. Tom, Fox, Trepel, and Poldrak (2007) investigate
the neural sources of loss aversion. Their goal was to identify and isolate the
regions of the brain associated with evaluating gambles and deciding whether or
not to accept them.

For each of the 16 subjects---eight males and eight females with an average
age of 22---three trials of the ``mixed gambles task'' are performed. The
\lstinline{ds005} data set contains the blood-oxygen-level-dependent contrasts
for each subject and trial. We successfully loaded several
\lstinline{bold.nii.gz} files, which have a shape of (64, 64, 34, 240).

Our goal is to reproduce Tom, Fox, Trepel, and Poldrak's findings based on
their whole-brain statistical analysis. The authors use two modeling approaches
for this---parametric and matrix analyses. We plan to investigate those
methods, but, given our limited exposure to them, will rely on simpler, though
equally valid, approaches. This will partly depend on the results from our
exploratory data analysis as well as on algorithms we're able to implement or
find Python packages for. Prior to any analysis, we will dedicate a substantial
portion of our efforts on preparing the data. We plan to use Lindquist's paper
on the statistical analysis of fMRI data\cite{lindquist} as a starting point
that will guide how we approach this and other aspects of our analysis.

\bibliography{proposal}

\end{document}
