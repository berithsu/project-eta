\documentclass[11pt]{article}

\usepackage[margin=0.75in]{geometry}

\title{The title of your project proposal}
\author{
  Jara, Jon\\
  \texttt{jonmigueljara}
  \and
  Sanderson, Will\\
  \texttt{wtsanderson}
  \and
  Shishido, Juan\\
  \texttt{juanshishido}
  \and
  Wu, Paul\\
  \texttt{yuanyiwu}
  \and
  Xu, Wendy\\
  \texttt{berithsu}
}

\bibliographystyle{siam}

\begin{document}
\maketitle

\abstract{You should have a short abstract.}

\section{Introduction}

Our research idea is based on the 2007 paper The Neural Basis of
Loss Aversion in Decision-Making Under Risk, written by Sabrina M. Tom, Craig
R. Fox, Christopher Trepel, and Russell A. Poldrack. Our goal is to reproduce
the major part of the paper with possible simplifications and also develop some
original analysis on the same data.

A significant portion of the paper is devoted to analyzing neural
indicators of loss aversion and correlating that neural loss aversion to
behavioral loss aversion. Test subjects are presented with gambles with equal
chances of winning and losing and different amounts of money to gain and lose.
These gamble offers and the subject’s decision of whether or not to accept the
gamble are recorded, as well as the subject’s neural activities during the
tasks. Behavioral risk aversion is measured through modeling the participant’s
decision on the amount of proposed gain and loss of the gambles. On the other
hand, neural risk aversion is measured through modeling the participant’s fMRI
data on the amount of proposed gain and loss of the gambles. By analyzing the
correlations between the coefficients of these models, the authors find that
some parts of the brain are particularly sensitive to loss and that neural
activities in the loss sensitive areas are closely related to the extent to
which a subject avoids risk in behavioral terms.

We will reproduce this part of the paper using mostly linear regression and
logistic models and evaluate our results against the paper’s results as well as
using error analysis. We will also visualize our research results consistently
with 2-D and possibly 3-D graphs. Besides, we would like to experiment with
Multi-Voxel Pattern Analysis on the fMRI data, which helps detect patterns of
brain activity by jointly analyzing multiple voxels at the same time. 

\section{Data}

\subsection{Description}

The data set includes 3 sample runs for each subject, with 16 subjects in
total. The subjects consist of 9 females and 7 males. For each run, the subject
is presented with a series of gambles with different combinations of gains and
losses, each combination randomly drawn from a gain/loss matrix of size 16 x
16. For the purpose of analysis, the data are collapsed into a 4 x 4 matrix.
The subject’s fMRI data generated during the tasks is available for each run.

\subsection{Acquiring}

A primary goal of ours is to ensure that individuals who wish to replicate or
improve upon our work can easily do so. To that end, in terms of data access,
we have created a bash script that fetches the data from www.openfmri.org and
untars it. It runs only if the data directory doesn't already exist. The script
also downloads the checksums and we've written a Python script to check that
the hashes match. We will incorporate the Python script in the Makefile for
downloading the data.

\section{Methods}

\subsection{Preprocessing}

We are considering several preprocessing techniques to use on our data.

\subsubsection{Smoothing}

Smoothing is a process that averages data---in this case, voxels---with its
neighbors. This is typically done using a Gaussian function. The selected width
of the distribution determines how much the data are smoothed.

There are several benefits related to smoothing. Below, we list the ones that
are relevant for our analysis.

\begin{itemize}
  \item increased sensitivity
  \item making the error distribution normal
\end{itemize}

We are also aware of the following disadvantages.

\begin{itemize}
  \item{reduced spatial resolution}
  \item{edge artifacts}
\end{itemize}

Typically, for single-subject analyses, a width of 4 mm is used.

%http://support.brainvoyager.com/functional-analysis-preparation/27-pre-processing/279-spatial-smoothing-in-preparation.html

\subsubsection{Principle Component Analysis}

We are also considering using principle component analysis (PCA) to reduce the
dimensionality of our data. We will investigate the standard PCA approach,
e.g., the one found in Scikit-Learn's decomposition submodule. Research by
Viviani, Gron, and Spitzer, however, has shown that \textit{functional}
principle component analysis is more effective than the standard approach. We 
plan to experiment with this approach.

\subsubsection{Motion Correction}

Another way in which we will attempt to increase the signal from our data is to
correct for motion. Nipy includes the function \texttt{FmriRealign4d} that
corrects for both motion and slice timing.

\subsection{Analysis}

\subsubsection{Primary}

The first task is to reproduce the results of the paper using our own tools.
The goal is to find areas of the brain in which neural loss aversion is
correlated to behavioral loss aversion. The result would lead to which parts of
the brain are responsible for making risky decision. The first step is to fit a
logistic regression on the behavioral data using a simple linear regression. We
used a combination of pandas and statsmodels packages to fit a logistic
regression model on the first subject and all the runs combined. Python modules
were used rather than R packages in order to keep the whole project in the same
language and for reproducibility. The logistic regression was modeled in this
manner:

\[ logit(p) = \beta_0 + \beta_1 X_{gain} + \beta_2 X_{loss} + \epsilon \] 

With gain and loss values beings the 2 regressors in the regression. Subject 1
had a gain to loss ratio of 2.33, which was inline with gain to loss ratio
cited by other sources in the paper (around 2.0). The paper was not specific in
how they dealt with multiple runs per subject but we decided that it would be
statistically sound to combine the runs for all the patients. In addition, the
paper conducted pre-processing steps like smoothing and motion correction that
we still need to employ. 

The logistic regression provided results inline with the paper (and past
``risk-averse'') studies and were in the end statistically significant,
however, the main goal of neurological study is to observe how these results
compare with neural signals and brain activity. This part of the analysis has
proven to be less trivial than a simple logistic regression. The challenges
stem mostly from working fMRI data in conjunction with the behavioral variables
``gain'' and ``loss''. We have tried using simple linear regression at each
voxel after convolving the fMRI data, but the results yielded blurry images of
brain slices the general locations of brain activity. While these locations
give a general idea of which parts of the brain are activated during runs of
the experiment, the resulting regressions leaves us with hundreds of beta
coefficients that represent activity without specifying whether the activity is
due to negative loss response and positive gain response. Further research is
needed to perform a similar conjunction analysis that was performed in the
paper. If the complexity of separating the different types of signals is beyond
the scope of our group, we will move on and use the simple regression model
with only one parameter. There is worry that without conjunction analysis we
will not see significant results like those found in the paper and it will be
hard to compare risk aversions to specific loss/gain values if we can
differentiate between the responses. Also, the paper specified specific regions
of the brain with large magnitudes of gain and loss response. To simplify our
analysis, we could restrict our study to whole brain analysis and see if we
find any significant correlation, otherwise we will attempt to segment the same
sections as the ones used in the original study. 

After modeling the fMRI data and hopefully getting valuable coefficients for
loss and gain responses, we will then have to use robust regression to model
the values of these neural response coefficients and the behavioral
coefficients. Robust regression was used in the paper as a way to decrease the
likelihood of any outliers. We might try to simplify this method by using
methods learned in class to remove outlier ourselves and then running a simple
linear regression. In the end, we believe we can reproduce the results of the
paper but we might need to simplify some of the methods, particularly in the
process of quantifying neural risk aversion and segmenting risk-averse areas of
the brain.

\subsubsection{Secondary}

\section{Results}

\section{Discussion}

\subsection{Challenges}

\subsubsection{Subsection}

Many of our issues so far have stemmed from the open ended nature of the
assignment, resulting in less direction than we are used to. This, combined
with the new git workflow, has given us some problems that we are working on
overcoming with a more rigid implementation plan that gives everyone a role.
We are currently using the workflow as best we can, but will be adding many
more Git issues in the coming days in order to better assign work and
facilitate multiple pull requests without merge conflicts. We're having good
success with Python code when we have a good implementation plan. However, a
good deal of our codebase deals with fMRI analysis that we haven't had much
experience working with. This has led to difficulty writing effective tests,
tests that can inspire full confidence in our implementations. We're working on
keeping them simple, using basic modular functions on small datasets to ensure
adequate coverage.

\subsubsection{Team}

We have been working well as a team, and most of our problems are simply
efficiency related. Each of us has applicable skills, we're just trying to get
them all streamlined. This is coming together through the workflow, mainly in
GitHub issues. This allows each of us to work on a different part of the
project at the same time. We have had trouble getting everyone together at the
same time due to erratic schedules, but have been overcoming this using the
workflow.

\subsection{Class Concepts}

We think that we could have used practice in more advanced fMRI analysis
techniques. We've been having some problems with preprocessing, which involves
much more detailed work than we thought. We're looking into some extra Python
libraries to help sort out some advanced preprocessing techniques, smoothing
out confounding variables like respiration and heartbeat. The workflow has been
the most helpful focus, but we think we could have used even more practice.
Because the workflow facilitates efficient team work, we would have liked to
have seen some more group exercises or homework aimed at perfecting the
process.

\bibliography{project}

\end{document}
